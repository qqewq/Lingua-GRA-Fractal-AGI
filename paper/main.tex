\documentclass[11pt]{article}

\usepackage[a4paper,margin=1in]{geometry}
\usepackage[T1]{fontenc}
\usepackage[utf8]{inputenc}
\usepackage[russian,english]{babel}
\usepackage{amsmath,amssymb,amsfonts}
\usepackage{bm}
\usepackage{hyperref}
\usepackage{graphicx}
\usepackage{enumitem}

\title{Lingua GRA:\\
Фрактально-осознанный научный язык для самоулучшающегося AGI\\[4pt]
{\large (Архитектура, реализация и эксперименты репозитория)}}

\author{AAA AAA}

\date{\today}

\begin{document}

\maketitle

\selectlanguage{russian}

\begin{abstract}
Lingua GRA — это экспериментальный языковой слой для AGI, в котором язык моделируется как иерархия гильбертовых пространств (символический, синтаксический, семантический, прагматический и мета-уровни) с функционалом ``пены'' GRA и дополнительным фрактальным регуляризатором. На уровне репозитория система реализована в виде ядра GRA (операторы-проекторы и многоуровневый функционал), набора нейросетевых энкодеров/декодеров, утилит для оценки корреляционной размерности эмбеддингов (алгоритм типа Грассбергера–Прокаччи), а также примеров: анализа фрактальной размерности word2vec-пространств и двухагентной среды Box World с возникающим языком. Эксперименты демонстрируют, как фрактальные профили естественных корпусов (сказки vs научные тексты) могут использоваться как явные целевые параметры для внутреннего языка AGI.
\end{abstract}

\tableofcontents

\section{Введение}

Базовые языковые модели (LLM) демонстрируют впечатляющие способности, но их внутренняя геометрия — эмбеддинги токенов, предложений и состояний — является побочным продуктом обучения, а не объектом явного контроля. Между тем эмпирические исследования embedding-пространств показывают, что естественные языки обладают выраженными фрактальными свойствами: мультифрактальной структурой, некратными целым intrinsic dimensions и устойчивыми диапазонами корреляционной размерности \(D_2\).[web:34][web:38][web:45][web:115]

Работы Ribeiro et al.\ демонстрируют, что облака word2vec-эмбеддингов для разных языков образуют мультифрактальные структуры, где локальная и глобальная фрактальные размерности различаются и коррелируют с филогенетическими и типологическими характеристиками языка.[web:34][web:38] Исследования intrinsic dimensions языковых фрактальных структур показывают, что для эмбеддингов n-грамм (английский и русский) наблюдаются устойчивые нецелые размерности порядка \(\sim 9\).[web:45][web:102] Анализ корреляционной размерности в статистическом многообразии (метрика Fisher–Rao) даёт универсальное значение \(\hat{\nu} \approx 6.5\) для нескольких языков, отличное как от шума, так и от простых графовых процессов.[web:105][web:41]

Lingua GRA трактует эти фрактальные характеристики не как любопытный побочный эффект, а как \emph{явную цель} для архитектуры языка AGI. В данном документе мы:

\begin{itemize}[leftmargin=1.2em]
  \item описываем многоуровневую архитектуру языка и GRA-функционал, реализованные в \texttt{gra\_core.py} и \texttt{language\_levels.py};
  \item вводим фрактальный слой и утилиты оценки корреляционной размерности (\texttt{fractal\_utils.py});
  \item описываем нейросетевую реализацию уровней и тренировочный цикл (\texttt{neural\_encoders.py}, \texttt{training.py});
  \item формулируем и реализуем эксперименты по фрактальной размерности (\texttt{fractal\_dim\_experiment.md}, \texttt{compute\_dimension.py}) и языку Box World (\texttt{box\_world\_language.md}, \texttt{simulate.py});
  \item очерчиваем контур мета-эволюции гиперпараметров (\texttt{meta\_evolution.py}).
\end{itemize}

\section{Архитектура репозитория}

Репозиторий организован следующим образом:
\begin{verbatim}
lingua_gra/
  __init__.py
  gra_core.py
  language_levels.py
  fractal_utils.py
  neural_encoders.py
  training.py
  meta_evolution.py
  utils.py
  compute_dimension.py

examples/
  fractal_dimension_analysis/
    README.md

box_world_language/
  simulate.py

simple_gra_demo.py
generate.py
config.yaml
architecture.md
fractal_background.md
neural_impl.md
fractal_dim_experiment.md
box_world_language.md
\end{verbatim}

Верхнеуровневые файлы \texttt{architecture.md}, \texttt{fractal\_background.md}, \texttt{neural\_impl.md} и \texttt{fractal\_dim\_experiment.md} содержат текстовые описания архитектуры уровней, фрактального фона, нейросетевой реализации и протоколов экспериментов соответственно.

\section{GRA-ядро и уровни языка}

\subsection{GRA-ядро (\texttt{gra\_core.py})}

Модуль \texttt{gra\_core.py} задаёт абстракции:

\begin{itemize}[leftmargin=1.2em]
  \item \textbf{HilbertSpace} — тонкая оболочка над пространством представлений \(\mathcal{H}\) с операциями скалярного произведения и квадрата нормы;
  \item \textbf{Projector} — нейросетевой проектор \(P_\theta\), аппроксимирующий оператор \(\mathcal{P}_{G_\ell}\) на уровне \(\ell\); пеной уровня выступает
  \[
  \Phi^{(\ell)}(\Psi^{(\ell)}, G_\ell) = \|(1 - \mathcal{P}_{G_\ell})\Psi^{(\ell)}\|^2;
  \]
  \item \textbf{GRAFunctional} — многоуровневый функционал
  \[
  J_{\text{GRA}}(\Psi) = \sum_{\ell} \Lambda_\ell \Phi^{(\ell)}(\Psi^{(\ell)}, G_\ell) + \text{(доп.\ термы)},
  \]
  реализованный как модуль, собирающий вклады по уровням и поддерживающий дополнительные регуляризаторы.
\end{itemize}

\subsection{Уровни языка (\texttt{language\_levels.py})}

Модуль \texttt{language\_levels.py} реализует иерархию уровней:
\begin{equation*}
\ell \in \{\text{SYMBOLIC}=0,\ \text{SYNTACTIC}=1,\ \text{SEMANTIC}=2,\ \text{PRAGMATIC}=3,\ \text{META}=4\}.
\end{equation*}

Каждый уровень представлен классом \textbf{BaseLevel}, включающим:

\begin{itemize}[leftmargin=1.2em]
  \item объект \texttt{HilbertSpace} с размерностью \(d_\ell\);
  \item энкодер (нейросетевой модуль) для перехода от входных данных к \(\Psi^{(\ell)}\);
  \item опциональный декодер;
  \item проектор \texttt{Projector}.
\end{itemize}

Конкретные классы:

\begin{itemize}[leftmargin=1.2em]
  \item \textbf{SymbolicLevel} — токены и базовый синтаксис: \texttt{Embedding + TransformerEncoder} и декодер в распределение по словарю;
  \item \textbf{SyntacticLevel} — заглушка для кодирования синтаксических деревьев (может быть заменена графовой сетью);
  \item \textbf{SemanticLevel} — семантические эмбеддинги предложений/документов;
  \item \textbf{PragmaticLevel} — представление состояния агента в среде (планы, цели);
  \item \textbf{MetaLevel} — мета-описания правил, конфигураций и архитектур.
\end{itemize}

Функция \texttt{build\_default\_levels} собирает стандартный набор уровней с разумными размерностями и весами \(\Lambda_\ell\).

\section{Фрактальный слой}

\subsection{Корреляционная размерность (\texttt{fractal\_utils.py})}

Модуль \texttt{fractal\_utils.py} реализует оценку корреляционной размерности \(D_2\) для облака эмбеддингов в \(\mathbb{R}^d\) по схеме типа Грассбергера–Прокаччи:[web:40][web:160]

\begin{enumerate}[leftmargin=1.5em]
  \item для набора точек \(X = \{x_i\}_{i=1}^N\) вычисляются попарные расстояния \(\|x_i - x_j\|\);
  \item задаётся диапазон радиусов \(r \in [r_{\min}, r_{\max}]\) (как доля от максимального расстояния);
  \item для каждого \(r\) считается корреляционный интеграл
  \[
  C(r) = \frac{2}{N(N-1)} \sum_{i<j} \Theta\big(r - \|x_i - x_j\|\big);
  \]
  \item строится зависимость \(\log C(r)\) от \(\log r\), и по линейной регрессии в рабочем диапазоне оценивается наклон \(D_2\).
\end{enumerate}

Эта процедура специально адаптирована для малых батчей эмбеддингов и может быть использована в тренировочном цикле.

\subsection{Фрактальный регуляризатор}

Фрактальный регуляризатор на уровне \(\ell\) задаётся как
\[
R_{\text{fract}}^{(\ell)}(\Psi^{(\ell)}) =
\Big(D_{2,\text{current}}^{(\ell)} - D_{2,\text{target}}^{(\ell)}\Big)^2,
\]
где \(D_{2,\text{current}}^{(\ell)}\) — оценка корреляционной размерности на текущем мини-батче эмбеддингов уровня \(\ell\), а \(D_{2,\text{target}}^{(\ell)}\) — целевая размерность, извлечённая из внешнего корпуса (например, научного текста) или заданная вручную.[web:38][web:115][web:118]

Полный функционал для уровня может быть записан как
\[
\tilde{\Phi}^{(\ell)}(\Psi^{(\ell)}, G_\ell) =
\|(1 - \mathcal{P}_{G_\ell}) \Psi^{(\ell)}\|^2
+ \gamma_\ell R_{\text{fract}}^{(\ell)}(\Psi^{(\ell)}),
\]
а глобальный функционал:
\[
J_{\text{fractal}}(\Psi) =
\sum_{\ell} \Lambda_\ell \tilde{\Phi}^{(\ell)}(\Psi^{(\ell)}, G_\ell).
\]

В коде это реализовано в \texttt{fractal\_utils.py} (\texttt{fractal\_regularizer}) и интегрировано в тренировочный цикл (\texttt{training.py}).

\section{Нейросетевая реализация и обучение}

\subsection{Энкодеры и декодеры (\texttt{neural\_encoders.py})}

Модуль \texttt{neural\_encoders.py} содержит базовые блоки:

\begin{itemize}[leftmargin=1.2em]
  \item \textbf{TokenEncoder/TokenDecoder} — символьный уровень: эмбеддинги токенов, позиционные эмбеддинги и \texttt{TransformerEncoder}, агрегирующий представление последовательности (по аналогии с \([CLS]\)-токеном).[web:162][web:166]
  \item \textbf{SentenceEncoder} — семантический энкодер, реализованный как MLP (заглушка, которую можно заменить LLM-энкодером).
  \item \textbf{PragmaticPolicy} — policy-сеть, принимающая наблюдение среды и эмбеддинг сообщения и выдающая logits действий (прагматический уровень).
\end{itemize}

Высокоуровневые классы \texttt{TokenLevelModel} и \texttt{SemanticLevelModel} связывают энкодер, проектор и (опционально) декодер в удобные модули.

\subsection{Тренировочный цикл (\texttt{training.py})}

Модуль \texttt{training.py} определяет:

\begin{itemize}[leftmargin=1.2em]
  \item класс \textbf{LinguaGRAModel}, объединяющий уровни, GRAFunctional и (опционально) прагматический policy;
  \item функцию \texttt{train\_step\_supervised\_token\_semantic}, которая демонстрирует комбинированную оптимизацию:
  \begin{itemize}
    \item реконструкция на символьном уровне (кросс-энтропийный loss),
    \item пена на символьном и семантическом уровнях,
    \item вклад GRA-функционала по уровням,
    \item опциональный фрактальный регуляризатор для семантического уровня;
  \end{itemize}
  \item функцию \texttt{policy\_gradient\_update} — набросок REINFORCE-обновления для прагматического уровня.[web:177][web:180]
\end{itemize}

\section{Эксперименты}

\subsection{Анализ фрактальной размерности (\texttt{examples/fractal\_dimension\_analysis})}

Пример в каталоге \texttt{examples/fractal\_dimension\_analysis} и скрипт \texttt{compute\_dimension.py} реализуют протокол:

\begin{enumerate}[leftmargin=1.5em]
  \item тренировку или загрузку word2vec-эмбеддингов для корпуса (сказки, научные тексты и т.п.);[web:202][web:204]
  \item выбор подмножества слов и извлечение их векторов;
  \item оценку корреляционной размерности \(D_2\) с помощью \texttt{correlation\_dimension};
  \item сохранение результата в JSON и последующее использование как \(D_{2,\text{target}}^{(2)}\) для семантического уровня.
\end{enumerate}

Отдельный протокол в \texttt{fractal\_dim\_experiment.md} описывает сравнение двух корпусов (сказки vs научные аннотации) и проверку гипотезы о различии их фрактальных профилей, опираясь на результаты Ribeiro et al.\ и работ по intrinsic/correlation dimensions.[web:34][web:38][web:45][web:105]

\subsection{Box World Language (\texttt{box\_world\_language.md}, \texttt{simulate.py})}

Эксперимент Box World реализует двухагентную среду:

\begin{itemize}[leftmargin=1.2em]
  \item агент Speaker видит полное состояние простого grid-world (агент, коробка, цель) и вырабатывает сообщение;
  \item агент Mover (прагматический уровень) видит локальное наблюдение и сообщение, выбирает действия для достижения цели;
  \item сообщения проходят через семантический уровень Lingua GRA, получая эмбеддинги \(\Psi^{(2)}\);
  \item policy Mover’а обучается методом policy gradient, а фрактальный регуляризатор для эмбеддингов сообщений подталкивает корреляционную размерность к заданному целевому \(D_2\).
\end{itemize}

Скрипт \texttt{simulate.py} логирует награды, длину эпизодов, policy-loss и оценку \(D_2\) для облака эмбеддингов сообщений по эпизоду. Это позволяет наблюдать, как фрактальный профиль языка коробки эволюционирует по мере обучения и как он меняется при включённом/выключенном фрактальном регуляризаторе.

\subsection{Простое GRA-демо и генерация (\texttt{simple\_gra\_demo.py}, \texttt{generate.py})}

\begin{itemize}[leftmargin=1.2em]
  \item \texttt{simple\_gra\_demo.py} — игрушечная тренировка символьного и семантического уровней на синтетических данных с GRA-пеной и фрактальным regularizer’ом.
  \item \texttt{generate.py} — пример генерации последовательности токенов с использованием обученной модели \texttt{LinguaGRAModel} и оценки \(D_2\) для семантических эмбеддингов сгенерированных текстов.
\end{itemize}

\section{Мета-эволюция (\texttt{meta\_evolution.py})}

Модуль \texttt{meta\_evolution.py} реализует простейшую схему мета-эволюции гиперпараметров:

\begin{itemize}[leftmargin=1.2em]
  \item \textbf{MetaState} хранит текущие веса \(\Lambda_\ell\), коэффициенты \(\gamma_\ell\) и целевые фрактальные размерности \(D_{2,\text{target}}^{(\ell)}\);
  \item \textbf{MetaController} вносит стохастические изменения (шаги в лог-пространстве для положительных параметров, аддитивные шаги для целевых размерностей);
  \item \textbf{MetaEvaluator} оценивает качество конфигурации по метрике ``фрактального выравнивания'' — среднему квадратичному отклонению оценённого \(D_2\) от целевых значений на валидационных эмбеддингах;
  \item функция \texttt{meta\_evolution\_step} реализует приём или отклонение шага по критерию улучшения метрики.
\end{itemize}

Эта схема может рассматриваться как зачаток мета-RL/гипероптимизации для настройки GRA и фрактального слоя.[web:182][web:186]

\section{Обсуждение и ограничения}

Lingua GRA в текущем виде — исследовательский прототип, предназначенный для экспериментов с фрактальными профилями языка и многоуровневыми представлениями, а не готовая производственная система. Многие компоненты (энкодеры, среды, мета-эволюция) реализованы в минимальном виде и требуют замены более мощными моделями и задачами. Оценка корреляционной размерности чувствительна к выбору диапазона масштабов и размерам выборки; для более серьёзного анализа могут потребоваться более устойчивые и быстрые алгоритмы.[web:40][web:276]

Тем не менее, уже в таком виде репозиторий даёт:

\begin{itemize}[leftmargin=1.2em]
  \item рабочий пайплайн от эмпирического измерения фрактальных параметров реального корпуса до внедрения этих параметров как целевых для внутреннего языка AGI;
  \item каркас многоуровневого GRA-функционала и фрактального регуляризатора;
  \item примеры сред (Box World), где можно изучать связь геометрии эмбеддингов и поведения агентов.
\end{itemize}

\section*{Заключение}

Мы описали архитектуру, реализацию и набор экспериментов репозитория Lingua GRA, в котором фрактальные характеристики естественного языка используются как явные цели при обучении многослойного языкового представления для AGI. В дальнейшем планируется:

\begin{itemize}[leftmargin=1.2em]
  \item использовать более крупные и разнообразные корпуса и модели (LLM-based encoders);
  \item перейти от одномерного параметра \(D_2\) к мультифрактальным спектрам;
  \item исследовать связь фрактального профиля с качеством рассуждений, устойчивостью к галлюцинациям и другими когнитивными свойствами AGI.
\end{itemize}

\end{document}
