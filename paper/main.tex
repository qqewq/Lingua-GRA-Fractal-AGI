\documentclass[11pt]{article}

\usepackage[a4paper,margin=1in]{geometry}
\usepackage[T1]{fontenc}
\usepackage[utf8]{inputenc}
\usepackage[russian,english]{babel}
\usepackage{amsmath,amssymb,amsfonts}
\usepackage{bm}
\usepackage{hyperref}
\usepackage{graphicx}
\usepackage{enumitem}

\title{Lingua GRA:\\
A Fractal-Aware Multi-Level Language for Self-Improving AGI\\[4pt]
Фрактально-осознанный многоуровневый язык для самоулучшающегося AGI}

\author{AAA AAA}

\date{\today}

\begin{document}

\maketitle

\begin{abstract}
\selectlanguage{english}
We present Lingua GRA, a research framework for a \emph{fractal-aware, multi-level language} tailored to Artificial General Intelligence (AGI). The core idea is to treat language as a hierarchy of Hilbert spaces (symbolic, syntactic, semantic, pragmatic, meta), each equipped with a GRA foam functional that measures deviation from level-specific goals, and to couple this with explicit fractal geometry targets for internal representations.[web:34][web:401] Empirical work on language embeddings shows robust self-similarity, long-range dependence, and non-integer intrinsic dimensions across languages and corpora; we turn these observations into differentiable objectives on the semantic level, using correlation dimension and related measures as regularizers.[web:34][web:434] On top of this core, we outline a ``multiverse'' platform of domain-specific languages Lingua-GRA-X (e.g., math, cinema), where linguistics acts as a universal glue and each domain X contributes its own structures and fractal profile.

\selectlanguage{russian}
Мы представляем Lingua GRA — исследовательский каркас для \emph{фрактально-осознанного многоуровневого языка}, ориентированного на искусственный общий интеллект (AGI). Язык моделируется как иерархия гильбертовых пространств (символический, синтаксический, семантический, прагматический, мета-уровни) с функционалом GRA-пены, измеряющим отклонение от целей каждого уровня, и явными фрактальными целями для внутренних представлений.[web:34][web:401] Эмпирические работы по языковым эмбеддингам показывают устойчивую самоподобность, длинную память и нецелые intrinsic dimensions для разных языков и корпусов; здесь эти свойства превращаются в дифференцируемые цели на семантическом уровне через корреляционную размерность и родственные меры.[web:34][web:434] Поверх этого ядра мы описываем «мультиверс» доменных языков Lingua-GRA-X (например, математика, кино), где лингвистика выступает универсальным клеем, а каждый домен X вносит свои структуры и фрактальный профиль.

\end{abstract}

\tableofcontents

\section{Introduction / Введение}

\selectlanguage{english}
Large language models (LLMs) have dramatically advanced the state of the art in text generation, reasoning, and coding, yet their internal languages remain emergent and weakly controlled. In most current systems, the geometry of token and sentence embeddings is a byproduct of next-token prediction, not an explicit design target.[web:434] At the same time, recent studies applying fractal analysis to language embeddings reveal self-similar, long-range dependent structures with non-integer intrinsic dimensions, fairly stable across languages and domains.[web:34][web:436] These results suggest that natural language occupies non-trivial fractal manifolds in high-dimensional spaces.

Lingua GRA explores what happens if we make these fractal signatures \emph{explicit objectives} inside a multi-level linguistic architecture for AGI. Concretely, we:
\begin{itemize}[leftmargin=1.5em]
  \item model each language level as a Hilbert space with a goal projector and a GRA foam functional;
  \item introduce a fractal regularizer on semantic embeddings, using correlation dimension as a differentiable proxy for intrinsic dimensionality;[web:34][web:401]
  \item provide a skeleton for domain-specific languages Lingua-GRA-X, where each domain (e.g., mathematics, cinema) contributes its own corpora, structures, and fractal targets.
\end{itemize}

\selectlanguage{russian}
Большие языковые модели (LLM) сильно продвинули генерацию текста, рассуждения и программирование, но их внутренний язык остаётся побочным продуктом обучения и слабо контролируется. Геометрия эмбеддингов токенов и предложений определяется задачей предсказания следующего токена, а не явными целями.[web:434] При этом работы по фрактальному анализу языковых эмбеддингов показывают самоподобные структуры с длинной зависимостью и нецелыми intrinsic dimensions, устойчивыми между языками и доменами.[web:34][web:436] Это говорит о том, что естественный язык живёт на нетривиальных фрактальных многообразиях в высоких размерностях.

Lingua GRA исследует, что получится, если сделать эти фрактальные подписи \emph{явными целями} внутри многоуровневой архитектуры языка для AGI. Конкретно:
\begin{itemize}[leftmargin=1.5em]
  \item каждый уровень языка моделируется как гильбертово пространство с проектором на цель и GRA-пеной;
  \item на семантическом уровне вводится фрактальный регуляризатор с корреляционной размерностью как дифференцируемым прокси intrinsic размера;[web:34][web:401]
  \item задаётся каркас доменных языков Lingua-GRA-X, где каждый домен (например, математика, кино) вносит свои корпуса, структуры и фрактальные таргеты.
\end{itemize}

\section{Core architecture / Базовая архитектура}

\subsection{Multi-level language as Hilbert hierarchy}

\selectlanguage{english}
We assume five main levels:
\[
\ell \in \{0,1,2,3,4\} =
\{\text{symbolic}, \text{syntactic}, \text{semantic}, \text{pragmatic}, \text{meta}\}.
\]
Each level \(\ell\) is associated with a Hilbert space \(\mathcal{H}_\ell\), a goal subset \(G_\ell \subset \mathcal{H}_\ell\), and a projector \(\mathcal{P}_{G_\ell}\). The GRA foam at level \(\ell\) is defined as
\[
\Phi^{(\ell)}(\Psi^{(\ell)}, G_\ell)
=
\big\|(1 - \mathcal{P}_{G_\ell})\Psi^{(\ell)}\big\|^2.
\]
The global GRA functional aggregates contributions across levels:
\[
J_{\text{GRA}}(\Psi) =
\sum_{\ell} \Lambda_\ell\, \Phi^{(\ell)}(\Psi^{(\ell)}, G_\ell),
\]
with non-negative weights \(\Lambda_\ell\) controlling the importance of each level.

\selectlanguage{russian}
Мы предполагаем пять основных уровней:
\[
\ell \in \{0,1,2,3,4\} =
\{\text{символический}, \text{синтаксический}, \text{семантический}, \text{прагматический}, \text{мета}\}.
\]
Каждому уровню \(\ell\) соответствует гильбертово пространство \(\mathcal{H}_\ell\), целевое подмножество \(G_\ell \subset \mathcal{H}_\ell\) и проектор \(\mathcal{P}_{G_\ell}\). GRA-пена на уровне \(\ell\) задаётся как
\[
\Phi^{(\ell)}(\Psi^{(\ell)}, G_\ell)
=
\big\|(1 - \mathcal{P}_{G_\ell})\Psi^{(\ell)}\big\|^2.
\]
Глобальный GRA-функционал агрегирует вклады по уровням:
\[
J_{\text{GRA}}(\Psi) =
\sum_{\ell} \Lambda_\ell\, \Phi^{(\ell)}(\Psi^{(\ell)}, G_\ell),
\]
где неотрицательные веса \(\Lambda_\ell\) задают важность каждого уровня.

\subsection{Fractal regularizer / Фрактальный регуляризатор}

\selectlanguage{english}
Following recent work on fractal patterns of language and intrinsic dimensions of embeddings, we approximate the correlation dimension \(D_2\) for mini-batches of semantic embeddings using a Grassberger–Procaccia-style estimator.[web:34][web:401] We then define a fractal regularizer:
\[
R_{\text{fract}}(\Psi^{(\text{sem})})
=
\big(D_{2,\text{current}} - D_{2,\text{target}}\big)^2,
\]
and augment the GRA functional at the semantic level:
\[
\tilde{\Phi}^{(\text{sem})} =
\Phi^{(\text{sem})} + \gamma\, R_{\text{fract}},
\]
with \(\gamma \ge 0\). This allows us to steer the geometry of semantic spaces toward target fractal profiles extracted from corpora (e.g., scientific text vs. fairy tales).[web:34][web:436]

\selectlanguage{russian}
Опираясь на работы о фрактальных паттернах языка и intrinsic dimensions эмбеддингов, мы приближённо оцениваем корреляционную размерность \(D_2\) для мини-батчей семантических эмбеддингов по алгоритму Грассбергера–Прокаччи.[web:34][web:401] Далее вводится фрактальный регуляризатор:
\[
R_{\text{fract}}(\Psi^{(\text{sem})})
=
\big(D_{2,\text{current}} - D_{2,\text{target}}\big)^2,
\]
и GRA-пена на семантическом уровне модифицируется:
\[
\tilde{\Phi}^{(\text{sem})} =
\Phi^{(\text{sem})} + \gamma\, R_{\text{fract}},
\]
где \(\gamma \ge 0\). Это позволяет подстраивать геометрию семантического пространства под целевые фрактальные профили, извлечённые из корпусов (например, научные тексты vs сказки).[web:34][web:436]

\section{Domain-specific languages Lingua-GRA-X / Доменные языки Lingua-GRA-X}

\selectlanguage{english}
On top of the core architecture, we define a family of domain-specific languages Lingua-GRA-X, each specified by:
\begin{itemize}[leftmargin=1.5em]
  \item a \textbf{DomainSpec} describing data, tasks, and structures of domain \(X\);
  \item a \textbf{LanguageConfig} listing active levels, GRA weights, and fractal targets;
  \item trained models implementing symbolic and semantic encoders and projectors.
\end{itemize}
Examples include Lingua-GRA-Math (linguistics + mathematics) and Lingua-GRA-Cinema (linguistics + film studies), where each domain contributes its own corpora and fractal profiles.

\selectlanguage{russian}
Поверх базовой архитектуры задаётся семейство доменных языков Lingua-GRA-X, каждый из которых определяется:
\begin{itemize}[leftmargin=1.5em]
  \item \textbf{DomainSpec} с описанием данных, задач и структур домена \(X\);
  \item \textbf{LanguageConfig} со списком уровней, весов GRA и фрактальных целей;
  \item обученными моделями символьных и семантических энкодеров и проекторов.
\end{itemize}
Примеры: Lingua-GRA-Math (лингвистика + математика) и Lingua-GRA-Cinema (лингвистика + кино), где каждый домен добавляет свои корпуса и фрактальные профили.

\section{Experiments and outlook / Эксперименты и перспективы}

\selectlanguage{english}
We outline two experimental tracks: (i) measuring and matching fractal profiles of different corpora (e.g., fairy tales vs. arXiv), and (ii) training agents in simple environments (e.g., Box World) whose communication channels are constrained by Lingua GRA with fractal regularization. Future work includes richer domains, multifractal spectra, and tighter links between fractal geometry and reasoning quality in AGI.[web:34][web:401][web:436]

\selectlanguage{russian}
Мы выделяем два направления экспериментов: (i) измерение и выравнивание фрактальных профилей разных корпусов (например, сказки vs arXiv), и (ii) обучение агентов в простых средах (например, Box World), где их канал коммуникации ограничен Lingua GRA с фрактальным регуляризатором. В будущем планируется подключение более сложных доменов, переход к мультифрактальным спектрам и исследование связи фрактальной геометрии с качеством рассуждений AGI.[web:34][web:401][web:436]

\end{document}
